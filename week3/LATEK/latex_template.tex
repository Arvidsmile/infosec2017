\documentclass[a4paper]{article}

%% Language and font encodings
\usepackage[english]{babel}
\usepackage[utf8x]{inputenc}
\usepackage[T1]{fontenc}

%% Sets page size and margins
\usepackage[a4paper,top=3cm,bottom=2cm,left=3cm,right=3cm,marginparwidth=1.75cm]{geometry}

%% Useful packages
\usepackage{amsmath}
\usepackage{graphicx}
\usepackage[colorinlistoftodos]{todonotes}
\usepackage[colorlinks=true, allcolors=blue]{hyperref}

% ------------------------
% Definitions for using Python (Added by Arvid)
\usepackage{color}
\definecolor{deepblue}{rgb}{0,0,0.5}
\definecolor{deepred}{rgb}{0.6,0,0}
\definecolor{deepgreen}{rgb}{0,0.5,0}

\usepackage{listings}

\usepackage{xcolor}
\usepackage{bera}% optional; just for the example

\lstset{
language=Python, backgroundcolor = \color{lightgray},
basicstyle=\ttfamily,
otherkeywords={self},
keywordstyle=\ttfamily\color{blue!90!black},
keywords=[2]{True,False},
keywords=[3]{ttk},
tabsize=4,
keywordstyle={[2]\ttfamily\color{yellow!80!orange}},
keywordstyle={[3]\ttfamily\color{red!80!orange}},
emph={MyClass,__init__},
emphstyle=\ttfamily\color{red!80!black},
stringstyle=\color{green!80!black},
showstringspaces=false
}
% -------------------------


\title{Information Secutiry - Week 3}
\author{Arvid Lindstrom - s2740761, Nil Stolt Anso - s2705338,\\ Razvan Andrei Poinaru - s2914751}

\begin{document}
\maketitle

% \begin{abstract}
% Your abstract.
% \end{abstract}
\abstract{Exercises submitted: 11, 12, 14, 15 and 17}
\section*{Exercise 11}

\subsection*{Program Output}

\begin{verbatim}
	formation Sec/infosecRepo/infosec2017/week3/ex11$ python repeatedSquares.py
	Base: 43210			# <-- user input
	Exponent: 23456
	Modulus: 99987
	Recursive solution = 82900
	Iterative solution = 82900
\end{verbatim}

\subsection*{Source of Program}

\begin{lstlisting}[language=Python]
from timeit import default_timer as timer

# Function used for reference since pythons
# pow() is using some optimizations
def naiveEponentiation(base, exponent):
	for i in range(exponent-1):
		base *= exponent
	return base

def repeatedSquaresIter(base, exponent, modulus):
	# Reference: Stamp's Book, page 99

	#1. Find exponent in binary
	binStr = bin(exponent)[2::] #skip the '0b'-part

	#2. Initialize values
	exponent = 0
	output = 0

	#3. Loop through binStr from MSB to LSB
	for bit in binStr:
		output = pow(pow(base, exponent), 2)
		#4. Calculate new exponent
		exponent = (exponent * 2)

		if(bit == '1'):
			output *= base
			exponent += 1
		output = (output % modulus)

	return output

def repeatedSquaresRec(base, exponent, modulus):
	# base case
	if(exponent == 1):
		return base % modulus
	# recursive step
	else:
		returned = repeatedSquaresRec(base, exponent / 2, modulus)
		returned = returned * returned % modulus
		if(exponent % 2 != 0):
			returned = returned * base % modulus
	return returned % modulus

base = int(raw_input("Base: "))
exp = int(raw_input("Exponent: "))
mod = int(raw_input("Modulus: "))

print "Recursive solution = " + \
	str(repeatedSquaresRec(base, exp, mod))
print "Iterative solution = " + \
	str(repeatedSquaresIter(base, exp, mod))
\end{lstlisting}

\section*{Exercise 12}

\subsection*{Answer to questions}

\begin{verbatim}
arvid@arvid-Aspire-V3-371:~/Desktop/Arvids Stuff/University/Year Four/
Block 1/Information Sec/infosecRepo/infosec2017/week3/ex12$ python ex12.py
Prime: 7919
[2, 37, 107]
7917 		# <-- modified to only show largest generator
arvid@arvid-Aspire-V3-371:~/Desktop/Arvids Stuff/University/Year Four/
Block 1/Information Sec/infosecRepo/infosec2017/week3/ex12$ python ex12.py
Prime: 23
[2, 11]
[5, 7, 10, 11, 14, 15, 17, 19, 20, 21]
\end{verbatim}
Generators of 23 are $[5, 7, 10, 11, 14, 15, 17, 19, 20, 21]$.\\
Larget generator of 7919 is 7917.

\subsection*{Source of Program}

\begin{lstlisting}[language=Python]
def checkIfPresent(c, factors):
#Determine if number is already present
	if len(factors) != 0:
		for i in range(len(factors)):
			if c == factors[i]:
				return factors
	factors.append(c)
	return factors

def prime_factors(n):
#Determine prime factors
	number = n
	i = 2
	factors = []
	while i*i <=n:
		if n%i:
			i+=1
		else:
			n //= i
			factors = checkIfPresent(i,factors)
	if n > 1:
		if n != number:
			factors = checkIfPresent(n,factors)
	return factors

def repeatedSquaresRec(base, exponent, modulus):
	# base case
	if(exponent == 1):
		return base % modulus
	# recursive step
	else:
		returned = repeatedSquaresRec(base, exponent / 2, modulus)
		returned = returned * returned % modulus
		if(exponent % 2 != 0):
			returned = returned * base % modulus
	return returned % modulus

def generators(pf, p):
	#Determine generators
	gener = []
	for g in range(2,p):
		flag = 0
		for x in pf:
			if repeatedSquaresRec(g, (p-1)/x, p) == 1:
				flag = 1
				break
		if flag == 0:
			gener.append(g)
	return gener

def main():
	prime = input("Prime: ")
	factors = []
	factors = prime_factors(prime-1)
	print factors
	gener = generators(factors,prime)
	print gener[-1] #<-- remove [-1] to show all generators

if __name__ == "__main__":
    main()
\end{lstlisting}

\section*{Exercise 14}

\subsection*{Answer to question regarding 57 * P1}
The number 57 can be represented in binary as: $111001$. Let the
least most significant bit $2^{0} : 1$ represent the point P1. Every
bit to the left (towards the most significant) will represent
$P_i = P_{i-1} + P_{i-1}$. Thus we perform one addition per bit needed
to represent the multiplier in binary. To yield the desired result we
perform addition on all points $P_i$ correlating with a high-bit. The
amount of bits required to represent 57 is 6. The first value is simply
P1 and thus requires no addition-operation. Therefore we have 5 additions
for bits ${2^1, 2^2, ... 2^5}$ and finally 3 additions since there are
4 high bits in the binary representation of 57 (and addition is a binary operation
taking two arguments). This yields 5 + 3 = 8 steps.

\subsection*{Program output}

\begin{verbatim}
Alice sends: (13, 16)
Bob sends: (7, 8)
--- Addition steps needed for 57 * P1: 8
Shared secret: (35, 20)
Alice and Bob will use the X-coord of (35, 20) : 35
Steps needed for: 209 * P(2, 7), a = 11, N = 167, -- 10
\end{verbatim}

\subsection*{Source of Program}

\begin{lstlisting}[language=Python]
from modinv import modinv

# Global used to track amount of addition-operations
numAdditions = 0

# Stamp's algorithm for adding two points
# P3(x3, y3) = P1(x1, y1) + P(x2, y2)
# -- Adding two points on an elliptic curve:

# Arguments: Two tuples p1 and p2 to be added
# Returns: One tuple p3 containing
# the new (x, y)-coordinates
def isEqual(p1, p2):
	if(p1[0] == p2[0] and p1[1] == p2[1]):
		return True
	return False

# Assuming a curve y^2 = x^3 + ax + b (% mod)
# p1 is the Pi with smallest Xi, p1[0] < p2[0]
def ellipticCurveAddition(p1, p2, a, mod):
	#0. Check the smallest Xi
	if (p1[0] > p2[0]):
		p1, p2 = p2, p1 # tuple swap (of tuples)

	#1. Calculate 'm'
	m = 0
	if(isEqual(p1, p2)):
		m = (3 * pow(p1[0], 2) + a) * \
			modinv(2 * p1[1], mod) % mod
	else:
		m = (p2[1] - p1[1]) * \
			modinv((p2[0] - p1[0]), mod) % mod

	x3 = 0
	y3 = 0
	# x3 = (m^2 - x1 - x2) % mod
	x3 = (pow(m, 2) - p1[0] - p2[0]) % mod
	# y3 = (m(x1 - x3) - y1) % mod
	y3 = ( m * (p1[0] - x3) - p1[1]) % mod

	return (x3, y3)

# "Naive multiplier"
def ellipticCurveMult(multiplier, point, a, modulus):
	increment = point
	for i in range(multiplier - 1):
		point = ellipticCurveAddition(point, increment, a, modulus)
	return point

# Find high bit points
# Given a multiplier and a point, compute all
# additions and return a list of the ones corresponding
# to the position with a high-bit in the multiplier
def findHighBitPoints(point, a, modulus, multiplier):
	global numAdditions
	binM = bin(multiplier)[2:]
	sumList = [0] * len(binM)
	sumList[0] = point
	orderedOutput = []

	# Starting with the least significant bit
	binM = list(reversed(binM))

	for i in range(1, len(binM)):
		prev = sumList[i-1]
		sumList[i] = ellipticCurveAddition(prev, prev, a, modulus)
		numAdditions += 1 #<-- track the amount of additions

	for i in range(len(binM)):
		if(binM[i] == '1'):
			orderedOutput.append(sumList[i])

	return orderedOutput

# Given a list of all points which had a high bit
# and recursively add them together like in hidden
# slide 51 from week 3
def recECCAdd(a, mod, list):
	global numAdditions
	if(len(list) == 2):
		numAdditions += 1 #<-- track the amount of additions
		return ellipticCurveAddition(list[0], list[1], a, mod)
	else:
		point = list[0]
		numAdditions += 1 #<-- track the amount of additions
		return ellipticCurveAddition(\
			recECCAdd(a, mod, list[1:]), point, a, mod)

# Fast multiplication using the bits of the multiplier
# Returns the point as a tuple
def fastECCMultiplier(point, a, modulus, multiplier):
	return recECCAdd(a, modulus, \
		findHighBitPoints(point, a, modulus, multiplier))

#### Answers to questions ####

a = 10
b = -21
p = 41 # modulus (prime)
P1 = (3, 6)

Alice_m = 44
Bob_m = 57

#Shared secret = 44 * (57 * P1), 57 * (44 * P1)

AliceMsg = fastECCMultiplier(P1, a, p, Alice_m)
print "Alice sends: " + str(AliceMsg)
numAdditions = 0 	#<-- reset the counter to track
					# necessary steps for 57 * P1
BobMsg = fastECCMultiplier(P1, a, p, Bob_m)
print "Bob sends: " + str(BobMsg)
print "--- Addition steps needed for 57 * P1: " + str(numAdditions)
sharedSecret = fastECCMultiplier(BobMsg, a, p, Alice_m)
print "Shared secret: " + str(sharedSecret)
print "Alice and Bob will use the X-coord of " + \
	str(sharedSecret) + " : " + str(sharedSecret[0])

# We test our counter against the given amount of
# steps from hidden slide 51
numAdditions = 0
fastECCMultiplier((2,7), 11, 167, 209)
print "Steps needed for: 209 * P(2, 7), a = 11, N = 167, -- " \
	+ str(numAdditions)
# ..and receive 10 steps as expected

\end{lstlisting}

\section*{Exercise 15}
This exercise uses the same code as exercise 14 with a slight
output modification shown in the Source-code subsection.

\subsection*{Values sent to TA}
\begin{verbatim}
3
5
157
4
9
19
135
\end{verbatim}

\subsection*{Values received from TA}

\begin{verbatim}
(79, 12)
\end{verbatim}

\subsection*{Source of Program}

\begin{lstlisting}[language=Python]
#### Answers to questions ####

# Public Key
a = 3
b = 5
mod = 157
MyPoint = (4, 9)

# Private key
m = 24

# MyPoint_m = (19, 135)
MyPoint_m = fastECCMultiplier(MyPoint, a, mod, m)

# Point sent from TA
TAPoint = (79, 12) # = n * (4, 9)

SharedSecret = fastECCMultiplier(TAPoint, a, mod, m)
print SharedSecret
\end{lstlisting}

\subsection*{Program output}
\begin{verbatim}
arvid@arvid-Aspire-V3-371:~/Desktop/Arvids Stuff/University/Year Four/
Block 1/Information Sec/infosecRepo/infosec2017/week3/ex15$ python ex15.py
(16, 99)
\end{verbatim}
The final shared point on the elliptic curve is:\\
(16, 99)

\section*{Exercise 17}

Here we once again use the code from exercise 14 with a modification shown below.\\
The final answer to k(reversed) * P = \\
$(367385334535545015949873084595410L, 171995391554293041834290054849881L)$

\subsection*{Program output}

\begin{verbatim}
arvid@arvid-Aspire-V3-371:~/Desktop/Arvids Stuff/University/Year Four/
Block 1/Information Sec/infosecRepo/infosec2017/week3/ex17$ python ex17.py

First we verify that our code works by using original 'k'
Result is :
(44646769697405861057630861884284L, 522968098895785888047540374779097L)
 and required 164 steps to compute.
Result is :
(367385334535545015949873084595410L, 171995391554293041834290054849881L)
 and required 165 steps to compute.
\end{verbatim}

\subsection*{Source of Program}

\begin{lstlisting}[language=Python]
numAdditions = 0

a = 321094768129147601892514872825668
b = 430782315140218274262276694323197
p = 564538252084441556247016902735257

Point = (97339010987059066523156133908935, \
		149670372846169285760682371978898)

k1 = 281183840311601949668207954530684
k2 = 486035459702866949106113048381182

print "First we verify that our code works by using original 'k'"
result = fastECCMultiplier(Point, a, p, k1)
print "Result is :\n" + str(result) + "\n and required " + \
	str(numAdditions) + " steps to compute."
numAdditions = 0
result = fastECCMultiplier(Point, a, p, k2)
print "Result is :\n" + str(result) + "\n and required " + \
	str(numAdditions) + " steps to compute."
\end{lstlisting}

% \section{Some examples to get started}

% \subsection{How to add Comments}

% Comments can be added to your project by clicking on the comment icon in the toolbar above. % * <john.hammersley@gmail.com> 2014-09-03T09:54:16.211Z:
% %
% % Here's an example comment!
% %
% To reply to a comment, simply click the reply button in the lower right corner of the comment, and you can close them when you're done.

% \subsection{How to include Figures}

% First you have to upload the image file from your computer using the upload link the project menu. Then use the includegraphics command to include it in your document. Use the figure environment and the caption command to add a number and a caption to your figure. See the code for Figure \ref{fig:frog} in this section for an example.

% \begin{figure}
% \centering
% \includegraphics[width=0.3\textwidth]{frog.jpg}
% \caption{\label{fig:frog}This frog was uploaded via the project menu.}
% \end{figure}

% \subsection{How to add Tables}

% Use the table and tabular commands for basic tables --- see Table~\ref{tab:widgets}, for example.

% \begin{table}
% \centering
% \begin{tabular}{l|r}
% Item & Quantity \\\hline
% Widgets & 42 \\
% Gadgets & 13
% \end{tabular}
% \caption{\label{tab:widgets}An example table.}
% \end{table}

% \subsection{How to write Mathematics}

% \LaTeX{} is great at typesetting mathematics. Let $X_1, X_2, \ldots, X_n$ be a sequence of independent and identically distributed random variables with $\text{E}[X_i] = \mu$ and $\text{Var}[X_i] = \sigma^2 < \infty$, and let
% \[S_n = \frac{X_1 + X_2 + \cdots + X_n}{n}
%       = \frac{1}{n}\sum_{i}^{n} X_i\]
% denote their mean. Then as $n$ approaches infinity, the random variables $\sqrt{n}(S_n - \mu)$ converge in distribution to a normal $\mathcal{N}(0, \sigma^2)$.


% \subsection{How to create Sections and Subsections}

% Use section and subsections to organize your document. Simply use the section and subsection buttons in the toolbar to create them, and we'll handle all the formatting and numbering automatically.

% \subsection{How to add Lists}

% You can make lists with automatic numbering \dots

% \begin{enumerate}
% \item Like this,
% \item and like this.
% \end{enumerate}
% \dots or bullet points \dots
% \begin{itemize}
% \item Like this,
% \item and like this.
% \end{itemize}

% \subsection{How to add Citations and a References List}

% You can upload a \verb|.bib| file containing your BibTeX entries, created with JabRef; or import your \href{https://www.overleaf.com/blog/184}{Mendeley}, CiteULike or Zotero library as a \verb|.bib| file. You can then cite entries from it, like this: \cite{greenwade93}. Just remember to specify a bibliography style, as well as the filename of the \verb|.bib|.

% You can find a \href{https://www.overleaf.com/help/97-how-to-include-a-bibliography-using-bibtex}{video tutorial here} to learn more about BibTeX.

% We hope you find Overleaf useful, and please let us know if you have any feedback using the help menu above --- or use the contact form at \url{https://www.overleaf.com/contact}!

% \bibliographystyle{alpha}
% \bibliography{sample}

\end{document}
