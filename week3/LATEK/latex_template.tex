\documentclass[a4paper]{article}

%% Language and font encodings
\usepackage[english]{babel}
\usepackage[utf8x]{inputenc}
\usepackage[T1]{fontenc}

%% Sets page size and margins
\usepackage[a4paper,top=3cm,bottom=2cm,left=3cm,right=3cm,marginparwidth=1.75cm]{geometry}

%% Useful packages
\usepackage{amsmath}
\usepackage{graphicx}
\usepackage[colorinlistoftodos]{todonotes}
\usepackage[colorlinks=true, allcolors=blue]{hyperref}

% ------------------------
% Definitions for using Python (Added by Arvid)
\usepackage{color}
\definecolor{deepblue}{rgb}{0,0,0.5}
\definecolor{deepred}{rgb}{0.6,0,0}
\definecolor{deepgreen}{rgb}{0,0.5,0}

\usepackage{listings}

\usepackage{xcolor}
\usepackage{bera}% optional; just for the example

\lstset{
language=Python, backgroundcolor = \color{lightgray},
basicstyle=\ttfamily,
otherkeywords={self},
keywordstyle=\ttfamily\color{blue!90!black},
keywords=[2]{True,False},
keywords=[3]{ttk},
tabsize=4,
keywordstyle={[2]\ttfamily\color{yellow!80!orange}},
keywordstyle={[3]\ttfamily\color{red!80!orange}},
emph={MyClass,__init__},
emphstyle=\ttfamily\color{red!80!black},
stringstyle=\color{green!80!black},
showstringspaces=false
}
% -------------------------


\title{Information Secutiry - Week 3}
\author{Arvid Lindstrom - s2740761, Nil Stolt Anso - s2705338,\\ Razvan Andrei Poinaru - s2914751}

\begin{document}
\maketitle

% \begin{abstract}
% Your abstract.
% \end{abstract}
\abstract{Exercises submitted: 11, 12, 14, 15 and 17}
\section*{Exercise 11}

\subsection*{Program Output}

\begin{verbatim}
	formation Sec/infosecRepo/infosec2017/week3/ex11$ python repeatedSquares.py
	Base: 43210			# <-- user input
	Exponent: 23456
	Modulus: 99987
	Recursive solution = 82900
	Iterative solution = 82900
\end{verbatim}

\subsection*{Source of Program}

\begin{lstlisting}[language=Python]
from timeit import default_timer as timer

# Function used for reference since pythons
# pow() is using some optimizations
def naiveEponentiation(base, exponent):
	for i in range(exponent-1):
		base *= exponent
	return base

def repeatedSquaresIter(base, exponent, modulus):
	# Reference: Stamp's Book, page 99

	#1. Find exponent in binary
	binStr = bin(exponent)[2::] #skip the '0b'-part

	#2. Initialize values
	exponent = 0
	output = 0

	#3. Loop through binStr from MSB to LSB
	for bit in binStr:
		output = pow(pow(base, exponent), 2)
		#4. Calculate new exponent
		exponent = (exponent * 2)

		if(bit == '1'):
			output *= base
			exponent += 1
		output = (output % modulus)

	return output

def repeatedSquaresRec(base, exponent, modulus):
	# base case
	if(exponent == 1):
		return base % modulus
	# recursive step
	else:
		returned = repeatedSquaresRec(base, exponent / 2, modulus)
		returned = returned * returned % modulus
		if(exponent % 2 != 0):
			returned = returned * base % modulus
	return returned % modulus

base = int(raw_input("Base: "))
exp = int(raw_input("Exponent: "))
mod = int(raw_input("Modulus: "))

print "Recursive solution = " + \
	str(repeatedSquaresRec(base, exp, mod))
print "Iterative solution = " + \
	str(repeatedSquaresIter(base, exp, mod))
\end{lstlisting}

\section*{Exercise 12}

\subsection*{Key used}
The key:
uoieazyxwvtsrqpnmlkjhgfdcb\newline
Maps onto: \newline
abcdefghijklmnopqrstuvwxyz\newline

\subsection*{Decrypted text}

\begin{verbatim}
9 common security awareness mistakes (and how to fix them)

To err is human, but to err in cyber security can cause major damage to an
organization. It will never be possible to be perfect, but major improvement
is possible, just by being aware of some of the most common mistakes and their
\end{verbatim}

\subsection*{Source of Program}

\begin{lstlisting}[language=Python]
#!/usr/bin/python

import sys

def readFile(name):
	file = open(name,'r')
	return file.read()

\end{lstlisting}

\section*{Exercise 14}

\subsection*{Key used}
The key:
uoieazyxwvtsrqpnmlkjhgfdcb\newline
Maps onto: \newline
abcdefghijklmnopqrstuvwxyz\newline

\subsection*{Decrypted text}

\begin{verbatim}
9 common security awareness mistakes (and how to fix them)

To err is human, but to err in cyber security can cause major damage to an
organization. It will never be possible to be perfect, but major improvement
is possible, just by being aware of some of the most common mistakes and their
\end{verbatim}

\subsection*{Source of Program}

\begin{lstlisting}[language=Python]
#!/usr/bin/python

import sys

def readFile(name):
	file = open(name,'r')
	return file.read()

\end{lstlisting}

\section*{Exercise 15}

\subsection*{Key used}
The key:
uoieazyxwvtsrqpnmlkjhgfdcb\newline
Maps onto: \newline
abcdefghijklmnopqrstuvwxyz\newline

\subsection*{Decrypted text}

\begin{verbatim}
9 common security awareness mistakes (and how to fix them)

To err is human, but to err in cyber security can cause major damage to an
organization. It will never be possible to be perfect, but major improvement
is possible, just by being aware of some of the most common mistakes and their
\end{verbatim}

\subsection*{Source of Program}

\begin{lstlisting}[language=Python]
#!/usr/bin/python

import sys

def readFile(name):
	file = open(name,'r')
	return file.read()

\end{lstlisting}

\section*{Exercise 17}

\subsection*{Key used}
The key:
uoieazyxwvtsrqpnmlkjhgfdcb\newline
Maps onto: \newline
abcdefghijklmnopqrstuvwxyz\newline

\subsection*{Decrypted text}

\begin{verbatim}
9 common security awareness mistakes (and how to fix them)

To err is human, but to err in cyber security can cause major damage to an
organization. It will never be possible to be perfect, but major improvement
is possible, just by being aware of some of the most common mistakes and their
\end{verbatim}

\subsection*{Source of Program}

\begin{lstlisting}[language=Python]
#!/usr/bin/python

import sys

def readFile(name):
	file = open(name,'r')
	return file.read()

\end{lstlisting}

% \section{Some examples to get started}

% \subsection{How to add Comments}

% Comments can be added to your project by clicking on the comment icon in the toolbar above. % * <john.hammersley@gmail.com> 2014-09-03T09:54:16.211Z:
% %
% % Here's an example comment!
% %
% To reply to a comment, simply click the reply button in the lower right corner of the comment, and you can close them when you're done.

% \subsection{How to include Figures}

% First you have to upload the image file from your computer using the upload link the project menu. Then use the includegraphics command to include it in your document. Use the figure environment and the caption command to add a number and a caption to your figure. See the code for Figure \ref{fig:frog} in this section for an example.

% \begin{figure}
% \centering
% \includegraphics[width=0.3\textwidth]{frog.jpg}
% \caption{\label{fig:frog}This frog was uploaded via the project menu.}
% \end{figure}

% \subsection{How to add Tables}

% Use the table and tabular commands for basic tables --- see Table~\ref{tab:widgets}, for example.

% \begin{table}
% \centering
% \begin{tabular}{l|r}
% Item & Quantity \\\hline
% Widgets & 42 \\
% Gadgets & 13
% \end{tabular}
% \caption{\label{tab:widgets}An example table.}
% \end{table}

% \subsection{How to write Mathematics}

% \LaTeX{} is great at typesetting mathematics. Let $X_1, X_2, \ldots, X_n$ be a sequence of independent and identically distributed random variables with $\text{E}[X_i] = \mu$ and $\text{Var}[X_i] = \sigma^2 < \infty$, and let
% \[S_n = \frac{X_1 + X_2 + \cdots + X_n}{n}
%       = \frac{1}{n}\sum_{i}^{n} X_i\]
% denote their mean. Then as $n$ approaches infinity, the random variables $\sqrt{n}(S_n - \mu)$ converge in distribution to a normal $\mathcal{N}(0, \sigma^2)$.


% \subsection{How to create Sections and Subsections}

% Use section and subsections to organize your document. Simply use the section and subsection buttons in the toolbar to create them, and we'll handle all the formatting and numbering automatically.

% \subsection{How to add Lists}

% You can make lists with automatic numbering \dots

% \begin{enumerate}
% \item Like this,
% \item and like this.
% \end{enumerate}
% \dots or bullet points \dots
% \begin{itemize}
% \item Like this,
% \item and like this.
% \end{itemize}

% \subsection{How to add Citations and a References List}

% You can upload a \verb|.bib| file containing your BibTeX entries, created with JabRef; or import your \href{https://www.overleaf.com/blog/184}{Mendeley}, CiteULike or Zotero library as a \verb|.bib| file. You can then cite entries from it, like this: \cite{greenwade93}. Just remember to specify a bibliography style, as well as the filename of the \verb|.bib|.

% You can find a \href{https://www.overleaf.com/help/97-how-to-include-a-bibliography-using-bibtex}{video tutorial here} to learn more about BibTeX.

% We hope you find Overleaf useful, and please let us know if you have any feedback using the help menu above --- or use the contact form at \url{https://www.overleaf.com/contact}!

% \bibliographystyle{alpha}
% \bibliography{sample}

\end{document}
